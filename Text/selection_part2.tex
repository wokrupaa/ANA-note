\subsection{Multiple candidates selection}
\label{subsec::multi}

Reconstruction of events may result in more than one signal candidate for the same \Bs events. It is generally associated with a number of candidates for a photon and combinations of \Ds meson daughters. The multiplication factor was calculated after each stage of selection to ensure that at the end, no more than one candidate per event left. In the table Tab. \ref{tab:multi2015raw}-\ref{tab:multi2018sel} there are multiple candidates rate for stripping and preselected, Run 2 sample (Appendix \ref{app::A}). Since no more than 1 or 2 events with no more than 2 candidates per event left after all selection criteria (Tab. \ref{tab:multiBDT}), no additional action has been taken. 

\begin{table}[h!]
\begin{center}
\begin{tabular}{ p{2.6cm}p{3.1cm}p{0.6cm} }
\hline
\hline
Multip.  & Number of events & \% \\
\hline
    \multicolumn{3}{c}{Run Downstream}\\
\hline

     1    & 1638	& 99,9 \\
     2    & 2	& 0,01 \\

\hline
\end{tabular}
\quad
\begin{tabular}{ p{2.6cm}p{3.1cm}p{0.6cm} }
\hline
\hline
Multip.  & Number of events & \% \\
\hline
    \multicolumn{3}{c}{Run 2 Long}\\
\hline

     1    & 1195	& 99,9 \\
     2    & 2	& 0,01 \\

\hline
\end{tabular}
\caption{Multiple candidate rate for Run 2 data sample after selection.}
\label{tab:multiBDT}
\end{center}
\end{table}%

\subsection{Selection of \Kstarm (construction works)}
\label{subsex::veto_non_reso}

In the analysis of \Bs\to\Dssp\Kstarm the non-resonant background is reduced by removing events in which the \KS\pion invariant mass is greater than 75 MeV from the nominal \Kstarm mass. The \Bs\to\Dssp\Kstarm is a pseudoscalar to vector-vector, decay, which means the \Kstarm  must be londitudially polarised, so the \KS helicity angle  follows a $cos^2\theta$ distribution *** Removing events with an absolute value of $cos^2\theta$ less than 0.3 improves the


\begin{table}[h!]
\begin{center}
\begin{tabular}{ p{3cm}p{3cm}p{5.5cm}}
\hline
\hline
& Description  & Requirement  \\
\hline
\Kstar  &  PIDK(\pion)     &   $<$ 0 \\
\hline
\end{tabular}
\caption{Offline selection requirements for \Kstar candidates.}
\label{tab:Reso_kstar_sel}
\end{center}
\end{table}%

 
\begin{table}[h!]
\centering
\begin{tabular}{p{3cm}p{4.5cm}p{6cm} }
\hline
\hline
 Variable  & Cut & Signal Efficiency (\%) \\
 \hline
  PIDK(\pion)  & $<$ 0 & 86.6 \\ 
 \hline
\end{tabular}
\caption{Efficiency of cuts (\Kstar candidates).}
\label{tab:Reso_kstar_eff}
\end{table}%